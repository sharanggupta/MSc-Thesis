%!TEX root = ../thesis.tex
%%%%%%%%%%%%%%%%%%%%%%%%%%%%%%%%%%%%%%%%%%%%%%%%%%%%%%%%%%%%%%%%%%%%%%%
%
% Literature Review
%
%%%%%%%%%%%%%%%%%%%%%%%%%%%%%%%%%%%%%%%%%%%%%%%%%%%%%%%%%%%%%%%%%%%%%%%
\chapter{Literature Review}%
\chaptermark{Literature Review}%
\label{chapter:literature_review}

\section{Cloud Computing}
Google cloud services and amazon web services have dominated the cloud computing space right from when it started gaining popularity. In August 2020, a report from Gartner[1] named both Google and Amazon in a group of 5 public cloud infrastructure providers that make up 80% of the IaaS market. We can draw many conclusions based on their feature by feature comparison, which would help us make a more informed decision. 

\subsection{Function as a Service}

At the heart of serverless, lies FaaS, or Function as a Service. In essence, FaaS allows us to borrow compute time with a millisecond granularity, thus adhering to the pay for only what you use principle of serverless. Being such an essential feature of serverless, all the popular cloud service providers offer this functionality in their own way. The most popular ones used are AWS Lambda functions offered by Amazon and Google Cloud Functions offered by Google. Depending on the use case, one or the other may be more suitable to our requirements. We can compare these two on a number of parameters, with pricing being at the forefront.

\bigskip

\begin{enumerate}[label=\Alph*.]
  \item AWS v/s GCP Pricing comparison \\ 
    \begin{longtable}{p{4cm}|p{3cm}|p{3cm}}
    \caption[AWS v/s GCP Pricing comparison]{AWS v/s GCP Pricing comparison} % 
    \label{tab:r1_r2_of_45} \\ 
    Metric & \column{AWS} & \column{GCP} \\ \hline
    Free Monthly Duration (GB-seconds) &
    \row{\linewidth}{400,000} & 
    \row{\linewidth}{400,000} \\ \hline
    %---%
    Free Monthly Requests &
    \row{\linewidth}{1 Million} & 
    \row{\linewidth}{2 Million} \\ \hline
    %---%
    Cost of Each Additional 1 Million  Requests &
    \row{\linewidth}\$0.20 & 
    \row{\linewidth}\$0.40 \\ \hline
    %---%
    Cost of Each Additional 1 GB-second &
    \row{\linewidth}\$0.000016 & 
    \row{\linewidth}\$0.0000125 \\ \hline
    %---%
    Duration granularity &
    \row{\linewidth}{1ms} & 
    \row{\linewidth}{100ms} \\ 
    %---%
    \end{longtable}
    From the above pricing comparison, we can infer that while GCP is very generous with its free tier monthly requests, the cost of additional 1 million requests is double of that of AWS, whereas that of an extra 1GB-second is lower. For large scales, where we expect to exceed the free tier over vehemently, the additional cost for each million requests is half of that of GCP for AWS. Depending on the GB-second requirement, the right choice at scale varies on the memory and time requirements. For memory intensive tasks, which could also be time consuming, GCP is the more economical choice, whereas for lighter, numerous tasks, AWS shines. For our use case, we use cloud functions to fetch calorie data from an API depending on the recipe ingredients, and hence, being a light task, AWS could be a more viable choice at scale. However, for a smaller number of users and recipes, within the free tier, GCP would be a better option.
    \\
  \item AWS v/s GCP Scalability \\ \\
    All serverless functions are built with scalability in mind, and as such, there is no difference in this aspect for GCP and AWS. Both of them use containers in the background to support FaaS, and hence, can easily spawn more or less containers depending on the workload automatically based on a script.
    \\
  \item AWS v/s GCP Concurrency \\ \\
    Concurrency for serverless functions dictates how a second request is processed by a FaaS instance while an earlier request is already being processed. In the background, it works by spawning another container the moment a concurrent request is sent. AWS lambda offers a soft limit of 1000 simultaneous requests in a region, with an option to reserve or provision more beforehand.
    \bigskip
    GCP cloud functions have no advertised concurrency limit. Thus, as we can see, AWS provides us with customised concurrency management options, whereas GCP is vague on how concurrent requests are handled.
    \\

\subsection{Database as a Service}
DataBase as a Service(DBaaS) also known as managed database service, is a cloud computing service that lets users access and use a cloud database system without purchasing and setting up their own hardware, installing their own database software, or managing the database themselves. The cloud provider takes care of everything from periodic upgrades to backups to ensuring that the database system remains available and secure 24/7. All the popular cloud service providers offer their own managed service for the same, with popular ones being DynamoDB from Amazon and Firebase from Google. We can compare the two on a number of parameters in order to contrast their use cases.

\bigskip

\begin{enumerate}[label=\arabic*.]
  \item DynamoDB v/s Firebase Pricing comparison \\ 
    \begin{longtable}{p{4cm}|p{4cm}|p{4cm}}
    \caption[DynamoDB v/s Firebase Pricing comparison]{DynamoDB v/s Firebase Pricing comparison} % 
    \label{tab:r1_r2_of_45} \\ 
    Metric & \column{AWS DynamoDB} & \column{GCP Firestore} \\ \hline
    Free Tier storage &
    \row{\linewidth}{25 GB per month} & 
    \row{\linewidth}{1 GB per day} \\ \hline
    %---%
    Free Tier Reads &
    \row{\linewidth}{25 Million read requests} & 
    \row{\linewidth}{50,000 read requests per day	} \\ \hline
    %---%
    Free Tier Writes &
    \row{\linewidth}{-} & 
    \row{\linewidth}{20,000 writes per day} \\ \hline
    %---%
    Free Tier Deletes &
    \row{\linewidth}{-} & 
    \row{\linewidth}{20,000 deletes per day} \\ \hline
    %---%
    Write Requests &
    \row{\linewidth}\$1.25 per million write request units & 
    \row{\linewidth}\$1.08 per million write requests \\ \hline
    %---%
    Read Requests &
    \row{\linewidth}\$0.25 per million read request units & 
    \row{\linewidth}\$0.36 per million read request units \\ \hline
    %---%
    Storage &
    \row{\linewidth}\$0.25 per GB-month & 
    \row{\linewidth}\$0.108 per GB-month \\ 
    \end{longtable}
    From the above pricing comparison, we can draw 2 major conclusions:

    \begin{enumerate}[label=\arabic*.]
      \item AWS DynamoDB is cheaper for read requests, but more expensive for writes
      \item GCP Firestore is cheaper for data storage
    \end{enumerate}
    \\
  \item DynamoDB v/s Firebase Performance comparison \\ \\
    While comparing insertion performance for DynamoDB and Firestore is difficult due to its differences, we can compare the retrieval performance. Firestore, while nifty for querying, falls short with respect to querying as compared to DynamoDb coupled with a GraphQl API, which allows for very specific optimised queries.
    \\
  \item DynamoDB v/s FireBase Fault Tolerance comparison \\ \\
    All the data is stored on solid-state disks (SSDs) and is \textbf{automatically replicated across multiple Availability Zones} in an AWS Region, providing built-in high availability and data \textbf{durability}[6].
    \bigskip
    While firestore does not explicitly mention any fault tolerance mechanism, they have stated certain limitations, within which we must operate[7].
    \\
\end{enumerate}

\subsection{Storage as a Service}

Storage as a Service or STaaS is cloud storage that you rent from a Cloud Service Provider (CSP) and that provides basic ways to access that storage. Enterprises, small and medium businesses, home offices, and individuals can use the cloud for multimedia storage, data repositories, data backup and recovery, and disaster recovery. There are also higher-tier managed services that build on top of STaaS, such as Database as a Service, in which you can write data into tables that are hosted through CSP resources.

\bigskip


The key benefit to STaaS is that you are offloading the cost and effort to manage data storage infrastructure and technology to a third-party CSP. This makes it much more effective to scale up storage resources without investing in new hardware or taking on configuration costs. You can also respond to changing market conditions faster. With just a few clicks you can rent terabytes or more of storage, and you don’t have to spin up new storage appliances on your own.
\bigskip

\begin{longtable}{p{4cm}|p{4cm}|p{4cm}}
\caption[Amazon Simple Storage Service v/s Google Cloud Storage]{Amazon Simple Storage Service v/s Google Cloud Storage} % 
\label{tab:r1_r2_of_45} \\ 
Metric & \column{AWS S3} & \column{GCP Cloud Storage} \\ \hline
Cost of storage &
\row{\linewidth}\$0.021 per GB per month & 
\row{\linewidth}\$0.020 per GB per month \\ \hline
%---%
Cost of Download &
\row{\linewidth}\$0.05 per GB & 
\row{\linewidth}\$0.12 per GB \\ \hline
%---%
Free Tier allowance &
\row{\linewidth}{-} & 
\row{\linewidth}{15 GB} \\ 
\end{longtable}

\bigskip

Comparing the two offerings from Google and Amazon, we can see that while the cost of storage for the two is comparable, with S3 being minutely more expensive, the cost of download is significantly cheaper. It is important to note that S3 has tiers of fault tolerance available, and the one mentioned is the cheapest offering.

\section{React v/s Angular v/s Vue}

Performance is one of the most important aspects to be considered for a front-end application. And when it comes to evaluating the performance of Angular, React and Vue, keep in mind that DOM is considered as the UI of any application. Both React and Angular take different approaches to update HTML files, but Vue has the best of both React and Angular frameworks. Let's get deep into Angular vs React vs Vue comparison:

\textbf{React}

\textbf{Pros:}

\begin{itemize}
  \item React is a front-end library that uses the \textbf{Virtual} DOM and enhances the performance of any size of application which needs regular content updates. For example, Instagram. 
  \item React is based on \textbf{single-direction data flow}. This will provide better control over the entire project.
  \item \textbf{Up to date factor}. Facebook team supports the library. Advice or code samples can be given by Facebook community.Using React+ES6/7, application gets high-tech and is suitable for high load systems.
\end{itemize}

\textbf{Cons:}

\begin{itemize}
  \item Learning curve. Being not full-featured framework it is requered in-depth knowledge for integration user interface free library into MVC framework.
  \item View-oriented is one of the cons of ReactJS. It should be found 'Model' and 'Controller' to resolve 'View' problem.Not using isomorphic approach to exploit application leads to search engines indexing problems.
  \item Lots of developers dislike JSX React's documentation, manuals are difficult for newcomers' understanding. React's large size library.
\end{itemize}

\textbf{Angular}

\textbf{Pros:}

\begin{itemize}
  \item MVC architecture allows Angular to \textbf{split tasks into logical chunks, reducing the initial load time} of a web pages. 
  \item The MVC also allows separation of concerns, with the view part being present on the client side, \textbf{drastically reducing queries in the background}.
\end{itemize}

\textbf{Cons:}

\begin{itemize}
  \item Due to the many features of this framework, sometimes they can create a burden for your projects, all translating into a heavier application and slower performance compared to React or Vue.  
  \item New, significant changes are introduced often. This can cause problems for developers when it comes to adapting to them.
\end{itemize}

\textbf{Vue}

\textbf{Pros:}

\begin{itemize}
  \item Vue makes development absolutely easy as the production-ready project weighs 20KB after min+gzip. That results in faster runtime and also stimulates development and \textbf{allows developers to separate template-to-virtual DOM from the compiler. More than that, when you have a minimum project size}, you don't need to put an extra effort over-optimization.
  \item One of the most important advantages of using Vue.js is its size as you can \textbf{get production-ready build project weighs just 20KB after min+gzip}. Size is unbeatable with all other frameworks such as Angular, ReactJS, and jQuery.
\end{itemize}

\textbf{Cons:}

\begin{itemize}
  \item Common plugins are useful as they work with various other tools to make development easy. Vue.js does not have most of the common plugins, and that is the drawbacks of Vue.js.
  \item Being a new member of a family, Vue has the smallest community support as compared to React and Angular. 
\end{itemize}

\textbf{Conclusion}

\textbf{React is scalable} - React plays a key role in the largest social media platform in the world—Facebook. It’s proven at massive scale and has a dedicated team of developers at Facebook with hundreds of other contributors outside the company.

\textbf{React is fast} - React uses intermediate representation (which they call "Virtual DOM") so that they can diff changes between different states, and make changes to a minimal amount of browser DOM nodes.

\section{GraphQL v/s  REST}

With \textbf{REST}, there’s a lot of back and forth and manual work. Calls can result in either over-fetching or under-fetching based on the API contract. 

Whereas \textbf{GraphQL} gets exactly the data you want on an API call. You have control over the query on a granular level, which is not something you can not easily do with REST since it’s not made for that specific purpose. Having this granular control will allow you to have fewer network calls and will require fewer developmental changes on client applications, since responsibility has been shifted to backends.

While REST offered many advantages to developers and became the de facto standard for businesses that deploy APIs, it also has a few disadvantages. These disadvantages arise from the fact that the server creates the representation of the resource, and the response to the client uses that. 

RESTful APIs often returned more data than what the client needed, alternatively, the client had to make multiple API calls to get all the data it needed. Developers also had to design the API endpoints keeping the front-end views in mind, and changes to the front-end views required changes to the API endpoint. 

\textbf{Conclusion} 

\textbf{GraphQL has faster product iterations on the frontend} - Developers can write queries specifying their data requirements, and the iterations for developing of the frontend can continue without having to change the backend. 

\textbf{GraphQL enables better analytics on the backend} - This enables the application owner to gain insights about which data elements are in demand, moreover, they will know which data elements aren't being used by clients anymore. 

\section{MaterialUI v/s Bootstrap v/s Ant design} 

\textbf{MaterialUI} is React Components that Implement Google's Material Design. 

\textbf{Bootstrap} is simple and flexible HTML, CSS, and JS for popular UI components and interactions. Bootstrap is the most popular HTML, CSS, and JS framework for developing responsive, mobile first projects on the web

\textbf{Ant Design} is a set of high-quality React components. An enterprise-class UI design language and React-based implementation. High-quality React components out of the box. 

\textbf{Conclusion}

\textbf{Ant Design} has powerful theme customization in every detail. It's written in TypeScript. 



